\documentclass{article}
\usepackage[scale=1.5]{ccicons}
\usepackage{hyperref}
\title{Franz Kafka's The Palace}
\author{Shad Gregory}
\date{}
\begin{document}
\maketitle
\begin{verse}
  \begin{center}
    \textbf{1} \\
  \end{center}
  When K. arrived, it was late in the night, \\
  The village was covered under deep snow, \\
  And the castle hill was nowhere in sight. \\
  Fog and darkness enveloped him and though \\
  The sky's faintest glow of light could not show \\
  The outline of the great Palace, K. stood \\
  On the wooden bridge, the darkness below, \\
  The seeming emptiness was understood \\
  To offer our hero the start of something good.
\end{verse}
\begin{verse}
  \begin{center}
    \textbf{2} \\
  \end{center}
  And then, a decisive intake of air, \\
  K. descended to the village below \\
  Hoping to find lodgings anywhere there. \\
  In the inn, \textit{bauren} still stirred to and fro \\
  In spite of the hour. There were no rooms, though \\
  The Innkeeper offered a sack of straw, \\
  And K.'s weariness had brought him so low \\
  That he eagerly accepted as fair \\
  The Innkeeper's offer to sleep on the floor there. \\
\end{verse}
\begin{verse}
  \begin{center}
    \textbf{3} \\
  \end{center}
  But soon then K. was roused from his slumber \\
  By a young man with a thespian's face. \\
  The \textit{bauren} were still there too in number \\
  And many had turned from their beer in case \\
  An entertaining spectacle took place. \\
  The young man was dressed in fancy city \\
  Clothes; his eyes were narrowed; it seemed the case \\
  The young man was the son of a pretty \\
  Big deal, and he was not trying to be witty!
\end{verse}
\newpage
\begin{verse}
  \begin{center}
    \textbf{4} \\
  \end{center}
  A big deal indeed, the son of the Palace \\
  Steward stood over K., his eyebrows strong, \\
  Ready to torment with polite malice \\
  Our poor K. over his excursion along \\
  The village's outskirts lurking among \\
  The shadows and darkness of the late hour. \\
  You have entered the village was his song, \\
  And the right to stay in any bower \\
  Or hut, resided only in Count Westwest's power.
\end{verse}
\begin{verse}
  \begin{center}
    \textbf{5} \\
  \end{center}
  Half sitting up and straightening his hair \\
  K. nonchalantly glanced up at the crowd;  \\
  The innkeeper, the \textit{bauren} in their chairs,  \\
  The young man asking should K. be allow'd, \\
  All waiting there for K. to speak out loud \\
  His intentions on such a snowy night. \\
  "Where am I?" K. asked as if he were proud \\
  of his ignorance, "I'm lost in the night." \\
  He cried, "Palace you say? But there was not one in sight."
\end{verse}
\begin{verse}
  \begin{center}
    \textbf{6} \\
  \end{center}
  The young man was astonished by K.'s act, \\
  "Why indeed, the Palace of Count Westwest!" \\
  "And you need the Count's permission, in fact, \\
  For a weary travellor to simply rest \\
  Overnight?" asked K. upon being press'd \\
  By the expecting crowd. Was it a dream \\
  That gave to him the notion that a guest \\
  Could be so cruelly turned out? It did seem \\
  To beggar belief. Such cruelty K. could not gleam.
\end{verse}
\begin{verse}
  \begin{center}
    \textbf{7} \\
  \end{center}
  "You must have permission!" was the reply. \\
  And with that, the dramatic young man turn'd \\
  To his audience, and said with a sigh, \\
  "Or maybe it's not required to have earn'd \\
  The Count's blessings!" And now having so learn'd \\
  The conditions of discretely dwelling \\
  overnight, grasping that which so concern'd \\
  the crowd, K. yawned, and perhaps overselling \\
  his nonchalance, announced his plans without yelling.
\end{verse}
\newpage
\begin{verse}
  \begin{center}
    \textbf{8} \\
  \end{center}
  "Now, if it is permission that I need," \\
  Said K. "Then it is permission I seek." \\
  And as if he were about to proceed, \\
  Cast off his blanket with nary a peek \\
  At the shocked crowd, barely able to speak. \\
  "Permission from whom?" sputtered the young man, \\
  "At this midnight hour?" he said with a shreik. \\
  "It isn't possible?" and K. began \\
  To yawn and stretch. "See, I like to sleep when I can!"
\end{verse}
\begin{verse}
  \begin{center}
    \textbf{9} \\
  \end{center}
  The young man was beside himself with rage, \\
  "Why you're not but a low-down dirty bum!" \\
  With a passion found only on the stage. \\
  "The count demands respect! Not some sass from \\
  A common tramp who's lower than pond scum! \\
  You must depart the count's territory \\
  At once!" At this, K. was able to drum \\
  Up the peace of a saint in God's glory, \\
  "Enough!" he said, and K. then began his story.
\end{verse}
\begin{verse}
  \begin{center}
    \textbf{10} \\
  \end{center}
  Does K. feel despair? Does he cry in the night? \\
  Is he so fixated on his mission \\
  That he no longer dreads the morning light? \\
  Why has he come here without permission? \\
  Travelled so far on this expedition \\
  Without a companion to help him through \\
  The snow and darkness with precision. \\
  Where is his family? Are they so few \\
  That K. was attracted to the castle in view?
\end{verse}
\begin{verse}
  \begin{center}
    \textbf{11} \\
  \end{center}
  "I've had enough of your nonsense." said K, \\
  "The Innkeeper and these good gentlemen \\
  Are my witnesses should I need to sway \\
  A jury of my peers. I take it then \\
  You would like to know why I am here in \\
  Your village. I am the land surveyor \\
  Sent for by the Count. Now there, you see when \\
  I saw the snow, layer upon layer, \\
  I sat out on the trek after a hopeful prayer.''
\end{verse}
\newpage
\begin{verse}
  \begin{center}
    \textbf{12} \\
  \end{center}
  ``But, unfortunately, I lost my way \\
  More than a few times and arrived so late \\
  That I knew it was too late in the day \\
  To report to the Castle in my state. \\
  This is why I chose to accept my fate \\
  And make do with camping out on the floor \\
  Here in the corner as much as I hate \\
  To give up the comforts of a locked door \\
  And a sweet bed, I knew my sleep would not be poor.''
\end{verse}
\begin{verse}
  \begin{center}
    \textbf{13} \\
  \end{center}
  "Tommorrow my assistants will arrive \\
  Via carriage with the equipment in tow. \\
  Now that's all that I'm willing to contrive \\
  As far as an explanation will go. \\
  Now goodnight fellas and, please, go pound snow!" \\
  K. turned to the stove and pulled his blanket tight. \\
  The Inn's mob retreated after K.'s show, \\
  Confused by this information's new light, \\
  They conversed in hushed tones while keeping K. in sight.
\end{verse}
\begin{verse}
  \begin{center}
    \textbf{14} \\
  \end{center}
  "Surveyor?" the word was tossed back and forth, \\
  And then a silence fell over the mob. \\
  The young man, eager to show off his worth, \\
  And now determined to finish the job, \\
  Whispered in a tone so as not to rob \\
  K. of his sleep but loud enough to hear \\
  "I'll call the Castle, ask about this slob, \\
  And check his story." he said with a sneer. \\
  He headed to the phone and brought it close to his ear.
\end{verse}
\begin{verse}
  \begin{center}
    \textbf{15} \\
  \end{center}
  "Good Goddamn!" thought K. to himself, "This place \\
  Is decked out to the nines! They have a phone?" \\
  Said telephone was crowded in a space \\
  Directly above K.'s head. In his own \\
  Weariness, among them, he was alone \\
  In overlooking the infernal thing. \\
  Now K.'s restful sleep was sure to be blown \\
  By the eager fellow's attempt to ring \\
  The Palace. And now poor K. had to hear him sing.
\end{verse}
\newpage
\begin{verse}
  \begin{center}
    \textbf{16} \\
  \end{center}
  Then the question was, would K. allow it? \\
  He decided to allow it, but now \\
  It was the case he could find no merit \\
  In feigning sleep, he flipped o'er with a scow \\
  And waited for the young man to find how \\
  To inquire without disrupting K.'s sleep. \\
  Across the way the dim light did allow \\
  K. to see the \textit{bauren} together deep \\
  In discussion and tightly piled in a heap.
\end{verse}
\begin{verse}
  \begin{center}
    \textbf{17} \\
  \end{center}
  K.'s arrival was no trivial news. \\
  Surveyors don't pop up every day! \\
  Every landlord had something to lose \\
  If the Count changed the lines any old way. \\
  The kitchen door was opened all the way, \\
  And its frame filled by the landlady's form. \\
  The host, eager to report on the fray, \\
  Tiptoed in her direction to inform \\
  The mighty Landlady of the incoming storm.
\end{verse}
\begin{verse}
  \begin{center}
    \textbf{18} \\
  \end{center}
  The telephone conversation began. \\
  The Palace Governor was sound asleep, \\
  But one of his lackeys was the night man, \\
  A certain Herr Fritz, who was known to keep \\
  Some abysmally late hours sometimes deep \\
  Into the night, was awake. The young man, \\
  Going by Schwarzer, proceeded to leap \\
  Into how he had found K., worn and wan, \\
  Sleeping on a dirty straw sack, so he began.
\end{verse}
\begin{verse}
  \begin{center}
    \textbf{19} \\
  \end{center}
  Of course, Schwarzer was suspicious of him! \\
  The landlord had neglected his duty; \\
  And so the burden was his to, with grim \\
  Determination, check out K.'s beauty \\
  Of a tale. Awakening K.'s booty \\
  From a deep sleep, his interrogation \\
  Of the man while he endured K.s snooty \\
  Attitude, and threw his accusation \\
  At K., along with expulsion from the nation.

\end{verse}

\newpage

\begin{verse}
  \begin{center}
    \textbf{20} \\
  \end{center}
  Schwarzer was shocked by K's ingratitude. \\
  Perhaps rightly so, since K. claimed to be \\
  A surveyor appointed to the good \\
  By the Count his very own self and we \\
  Can all assume that it's Scwarzer's duty \\
  To verify K.'s claim, and so for sure \\
  He was going to ask this Fritz to see \\
  Into K.'s claim, that the count did procure \\
  His services, and in his motives, were they pure?
\end{verse}

\begin{verse}
  \begin{center}
    \textbf{21} \\
  \end{center}
  And then all was quiet. The whole lot of them \\
  Waited with bated breath for Fritz to return \\
  With the Palace's answer, a precious gem \\
  Of information, so that they might learn \\
  Could they send K. flying out with a stern \\
  Flogging? And K.? He kept his poker face \\
  Firmly intact and determined to earn \\
  His spot with the rats in the Palace's race. \\
  He stayed stoic, steadfast; his mask firmly in place.
\end{verse}

\begin{verse}
  \begin{center}
    \textbf{22} \\
  \end{center}
  Soon the phone's bray cut through the quiet. \\
  It was Fritz, his report musta been brief \\
  For Schwarzer, as if to start a riot, \\
  Shouted, "I told you so!" in stark relief \\
  And slammed down the receiver, "A liar-in-chief! \\
  Nobody has heard of this surveyor!" \\
  K. dove under his blanket in the belief \\
  That Schwarzer, \textit{bauren}, and everyone there \\
  Would pounce upon poor K. right there without a care.
\end{verse}

\begin{verse}
  \begin{center}
    \textbf{23} \\
  \end{center}
  He waited for the assault with unease, \\
  Under the blanket K. said a small prayer, \\
  When the phone rang again it seemed to squeeze \\
  the soul out of poor K. as he lay there. \\
  He slowly poked his head out, with great care \\
  And then watched Schwarzer return to the phone, \\
  Meekly allowing the caller to share \\
  A long story that led Schwarzer to groan, \\
  "Mistake? How can I explain this all on my own?"
\end{verse}

\newpage

\begin{verse}
  \begin{center}
    \textbf{24} \\
  \end{center}
"This puts me in quite a pickle, you know, \\
And the office manager made the call?" \\
K. listened to this telephonic show \\
With great interest. Had the Palace all \\
but yielded to K. without even a brawl? \\
K. felt hope, saw light where there was no light \\
Before. Was the Palace totally in thrall \\
To its own power that it had lost sight \\
Of the freedom that K. would enjoy in the fight?
\end{verse}

\begin{verse}
  \begin{center}
    \textbf{25} \\
  \end{center}
  But then, K. began to reflect again, \\
  Perhaps the keen Palace had taken stock \\
  Of him and, finding him wanting, had then \\
  Likened him to Sisyphus and his rock, \\
  A sad figure racing against the clock, \\
  Suddenly the Palace had, with a smile, \\
  Taken up the struggle to perhaps block \\
  K.; His effort to join the village while \\
  Enjoying a Palace job and living in style.
\end{verse}

\begin{verse}
  \begin{center}
    \textbf{26} \\
  \end{center}
  Someone told falsehoods about Joseph K. \\
  And as he lay in a peaceful slumber, \\
  Strange men arrested him that faithful day. \\
  It was not his choice to pluck a number \\
  From a ticket dispenser and lumber \\
  To the back of the line. He did not choose \\
  To embrace the law that would encumber \\
  His efforts, leading Joseph K. to lose \\
  His own life to two men in fading twilight hues.
\end{verse}

\begin{verse}
  \begin{center}
    \textbf{27} \\
  \end{center}
  But, what can be said then of our poor K.? \\
  He certainly chose his peculiar fate, \\
  Groveling for a position all day, \\
  Every day, no matter how small or great. \\
  Sleeping away by the palace's gate \\
  And prostrating himself before the law \\
  Seasons will pass and K. will track the date \\
  Until he is robbed of all that he saw \\
  And his poor body is found in the springtime's thaw.
\end{verse}

\newpage

\begin{verse}
  \begin{center}
    \textbf{28} \\
  \end{center}
  Schwarzer approached K. bowing and scraping, \\
  But K. only shooed him off on his way. \\
  K. was offered a chance of escaping \\
  The pub for the innkeeper's room til day, \\
  But K. shook his head and would only say \\
  To bring a washbasin with soap and towel \\
  Items the landlady brought right away. \\
  The lamp was darkened, with a nary a growl \\
  From the \textit{bauren} as they left K. with the night owl.
\end{verse}

\begin{verse}
  \begin{center}
    \textbf{29} \\
  \end{center}
  K. slept so well that night there on the floor, \\
  Apart from a scurrying rat or two.\\
  Franz tells us not if he let out a snore, \\
  but only that he slept there so, so, true. \\
  Oh, and you should know that he was feed too. \\
  I'm sure the breakfast, though we are not told, \\
  Was solid and true. Payment was not due, \\
  said the Innkeeper, all food, hot or cold, \\
  Was to be paid for by the palace's good gold.
\end{verse}

\begin{verse}
  \begin{center}
    \textbf{30} \\
  \end{center}
  K. polished off his meal and wanted to go \\
  Straight to the village, but he took pity \\
  Upon the poor innkeeper, and then so \\
  Let the sad fellow sit down and pretty \\
  Much take a break. "I'm new to this city, \\
  And my knowledge of the Count is so poor!" \\
  Said K., "Is it true that he pays pretty \\
  Well for good work? I would be mighty sore \\
  If this weren't true, leaving my family and more."
\end{verse}

\begin{verse}
  \begin{center}
    \textbf{31} \\
  \end{center}
  Wow, K. with wife and child? This backstory \\
  will come to nought, dear reader, so calm down! \\
  We perch on a distant promontory, \\
  And can no longer see what was once known. \\
  Is K. lying? Did Kafka mess around \\
  And not finish his wonderful novel? \\
  Leaving another king without a crown, \\
  An idea left rotting in a hovel, \\
  Never to know, no matter how much we grovel.
\end{verse}

\newpage
\begin{verse}
  \begin{center}
    \textbf{32} \\
  \end{center}
  "I don't hear any gripes about the Lord," \\
  Said the innkeeper, "He pays fairly well." \\
  "I'm not afraid to put in a good word \\
  For myself!" said K. "I would like to sell \\
  My skills personally to the Herren." \\
  The \textit{Wirt} seemed to retreat into his shell \\
  As he perched anxiously on his barren \\
  Windowsill. What fear and terrors did dwell \\
  In the poor man's chest? Did he see Charon \\
  Waiting to ferry his soul across the Acheron?
\end{verse}

\begin{verse}
  \begin{center}
    \textbf{33} \\
  \end{center}
  The \textit{Wirt} there with his big brown anxious eyes, \\
  K. just had to distract him from his woes. \\
  "My assistants," said K. "I would surmise, \\
  Will soon be here. So, if the need arose, \\
  Can you provide rooms for them to repose?" \\
  "Certainly," he said, "but will they not stay \\
  With you at the Palace?" K. shrugged. "Who knows \\
  What work they have for me from day to day. \\
  You see, I may need to stay down here, who can say?"
\end{verse}

\begin{verse}
  \begin{center}
    \textbf{34} \\
  \end{center}
  "You don't know the Palace." was the reply \\
  The \textit{Wirt} gave him in an awed and hushed tone. \\
  K. considered this and then with a sigh \\
  Said, "My ego is big, but I can be shown \\
  the error and shortcomings of my own \\
  Beliefs. About the Palace, all I can say, \\
  Is that they know how to find (all alone!) \\
  The right land surveyor, but staying way \\
  Up there could have its advantages you might say.''
\end{verse}

\begin{verse}
  \begin{center}
    \textbf{35} \\
  \end{center}
  The \textit{Wirt} began to bite his lip in fear, \\
  And a pang of pity rose in K.'s breast. \\
  What would it take to give this man good cheer? \\
  K. rose from his seat; he had done his best \\
  To set the \textit{Wirt} at ease, but now the rest \\
  Of his day needed to begin. As he \\
  Was leaving, a portrait of a depressed \\
  Figure of a man on the wall caught the \\
  Corner of K.'s eye; he approached it so to see.
\end{verse}

\newpage
\begin{verse}
  \begin{center}
    \textbf{36} \\
  \end{center}
  The half-length portrait was of a spirit,\\
  'Bout fifty years of age, with a forehead\\
  So ponderous and so heavy that it\\
  Seemed to depress his chin, it could be said,\\
  Into his chest. His head was sunk so it  \\
  Was nigh impossible to see his eyes.\\
  His hand clutched his thick hair as if unfit\\
  To support a head that could not arise,\\
  Like Lazarus from the grave, to its former size.
\end{verse}

\begin{verse}
  \begin{center}
    \textbf{37} \\
  \end{center}
  K. stood in front of the gloomy picture. \\
  "Who is that?" asked K. "Is that the Count?" \\
  K. spoke aloud without guile or stricture, \\
  Not looking or taking into account \\
  The presence of the Wirt and the amount \\
  That could then be heard. The \textit{Wirt} shook his head \\
  "No," he said in a whisper, "Not the Count, \\
  But the Palace Governor." The \textit{Wirt}'s dread \\
  Was palpable in a way that could not be read.
\end{verse}

\begin{verse}
  \begin{center}
    \textbf{38} \\
  \end{center}
  "Such a handsome gov'nor in the Palace." \\
  Said K. "Pity 'bout his brat of a son!" \\
  The Wirt pulled K. closer without malice \\
  And whispered, "Schwarzer's full of a fuckton \\
  of shit! His father's a subgov'nor, one \\
  Of the lowest at that!" "The devil!" \\
  laughed K, but the Wirt didn't join in the fun. \\
  "Oh, His father is totally high-level!" \\
  In the paranoia, K began to revel. \\
\end{verse}

\begin{verse}
  \begin{center}
    \textbf{39} \\
  \end{center}
  "Here and there you see powerful people! \\
  How about me? Am I powerful too?" \\
  The Wirt said, "No you're one of the sheeple." \\
  "I can tell that you've got me pegged, that's true. \\
  But, you see, I'm not as honest as you, \\
  I respect them, I don't care to admit it." \\
  K. tapped him on his cheek in order to \\
  Gain his affection and he saw it \\
  Bring the faintest smile to his face for a visit.
\end{verse}
\newpage
\begin{verse}
  \begin{center}
    \textbf{40} \\
  \end{center}
  The smile transformed the \textit{Wirt} into a boy, \\
  What with his soft and almost beardless face. \\
  How did he find himself in spousal joy \\
  With an older corn-fed gal in this place? \\
  K. spied the \textit{Wirtin} through an open space \\
  In the kitchen with her elbows sticking \\
  Out and decided that he need not chase \\
  The matter any further. Heels clicking, \\
  K. set out to give the beautiful day a licking.
\end{verse}

\begin{verse}
  \begin{center}
    \textbf{41} \\
  \end{center}
  The Palace was framed against the clear air, \\
  And it loomed above the village below. \\
  It met K's expectations hanging there. \\
  It was not some musty old castle, oh \\
  No! Nor an ornate building just for show, \\
  But rather a jumble of squat buildings \\
  With only one tower; K. could not know \\
  If it was a church without the gilding. \\
  But alas! naming such things he was not skilled in.
\end{verse}

\begin{verse}
  \begin{center}
    \textbf{42} \\
  \end{center}
  K. soldiered on with his eyes fixed ahead \\
  On the Palace. But as our K. came near \\
  The sight of the Palace began to spread \\
  Out before him; he was struck by a fear. \\
  It was not grandiose vision, that was clear, \\
  Rather a miserable little town \\
  Assembled from stone houses in such a queer \\
  Manner. A wave of displeasure crashed down \\
  Upon him. How could this place command such renown?
\end{verse}

\begin{verse}
  \begin{center}
    \textbf{43} \\
  \end{center}
  K. stopped in front of a long one-storey \\
  Building hiding behind a fenced garden \\
  That was now covered in snowy glory. \\
  Just then a gang of kids and their warden \\
  Poured out of the building, beg your pardon, \\
  Left the school in an orderly manner. \\
  The children drew up short like a bard in \\
  Thought and crowded 'neath their teacher's banner, \\
  Staring at K. and loudly raising a clamor.
\end{verse}
\newpage
\begin{verse}
  \begin{center}
    \textbf{44} \\
  \end{center}
  The teacher was a young small man with slim \\
  Shoulders and an upright posture, but he \\
  Was not ridiculous, oh no, but trim. \\
  K. stood all alone, only he could be \\
  Seen. So, not to appear as out to sea, \\
  K. chose to speak first, "Good day, \textit{Herr} Teacher." \\
  An eerie silence settled upon the \\
  Group, like the calm called for by a preacher. \\
  The young man gestured towards a village feature.
\end{verse}

\begin{verse}
  \begin{center}
    \textbf{45} \\
  \end{center}
  "Are you looking at the Palace?" he said. \\
  "Yes," said K. "I only arrived last night. \\
  Went for a stroll and decided to head \\
  Out this cold morning to check out the sight \\
  of the Palace." Something was not all right, \\
  "You don't like the Palace?" the teacher asked \\
  Abruptly. At that K. found himself quite \\
  Taken aback. "What?" K. said to the last \\
  Question. But then K. gave his response a new cast.
\end{verse}
\begin{verse}
  \begin{center}
    \textbf{46} \\
  \end{center}
"Why do you think I don't like the Palace?" \\
K. asked. "No stranger likes it very much." \\
Was the reply. K. was full, like Alice \\
In \textit{Through the Looking Glass}, of questions such \\
As, "You must know the Count?" with just a touch \\
Of eagerness. "No!" the teacher replied \\
And turned to leave, but K. knocked his crutch \\
Away with "How? The Count is far and wide \\
Known everywhere, how can you stand to act so snide?"
\end{verse}
\begin{verse}
  \begin{center}
    \textbf{47} \\
  \end{center}
  "How should I know the Count?" cried the teacher. \\
  He gestured and said, "\textit{Enfants innocents}!" \\
  But discouragement was not a feature \\
  of K.'s core nature. K. asked, "Can I plan \\
  On paying you a visit my good man?" \\
  And quickly wrapped himself in self-pity. \\
  "I belong among the elite rather than \\
  The plebian \textit{bauren} and their gritty \\
  Ways. Nor the Palace and it's surrounding city."
\end{verse}
\newpage
\begin{verse}
  \begin{center}
    \textbf{48} \\
  \end{center}
  "The \textit{bauren} and the Palace are the same." \\
  The teacher said with a shrug. "That may be," \\
  K. replied, "But it doesn't change the game. \\
  Can I pay you a visit?" He said that he \\
  Lived on Swan Street, but it was without glee, \\
  More like a statement of fact than an invite. \\
  K. pressed ahead, "Good, I will come and see \\
  You." The teacher nodded in morning's light, \\
  And he and the kiddie gang were soon out of sight.
\end{verse}
\begin{verse}
  \begin{center}
    \textbf{49} \\
  \end{center}
  Suddenly the very wind left K.'s sails. \\
  He felt tired, real fatigue for the first time \\
  Since he arrived, treking through snow and gales, \\
  Calmly taking step by step the long climb \\
  To the village inn, needing neither rails \\
  Nor horse to accomplish his long journey. \\
  He needed someone to listen to his tales, \\
  But much like a sick man on a gurney, \\
  Each acquaintance was like a boring attorney.
\end{verse}
\begin{verse}
  \begin{center}
    \textbf{50} \\
  \end{center}
  K. so needed to reach the Palace gate, \\
  He started, but it was such a long way. \\
  How far could he go in his current state? \\
  The main road did not lead in any way \\
  to the Palace Hill, in the light of day \\
  It only seemed, to K.'s eyes, to come close \\
  Only to perversely turn just away, \\
  As if the road had an instinct to cross \\
  K.'s very intentions and guffaw at his loss.
\end{verse}
\begin{verse}
  \begin{center}
    \textbf{51} \\
  \end{center}
  The village was an unending parade \\
  Of tiny houses behind mounds of snow \\
  And frosted windows, while the streets displayed \\
  Their emptiness in an unearthly show. \\
  K. continued plodding on the road, though \\
  It did not turn back towards or away \\
  From the Palace. K. stopped. He could not know \\
  If the village road lead in any way \\
  To the Palace, and he then turned from the roadway.
\end{verse}
\newpage
\begin{verse}
  \begin{center}
    \textbf{52} \\
  \end{center}
  Down a narrow alley he turned, his feet \\
  Sinking deeper into the piles of snow. \\
  Sweat broke out and he could feel his heart beat \\
  Against his chest. Then K. stopped and could go \\
  No further. His exhaustion had brought him low. \\
  Why had he left the safety of the main \\
  Road? Now to go out like some common joe? \\
  But he was not lost on some desert plain, \\
  For \textit{bauren} cottages lined both sides of the lane.
\end{verse}
\begin{verse}
  \begin{center}
    \textbf{53} \\
  \end{center}
  K. knelt down low and fashioned a snowball. \\
  He took careful aim and threw it up side \\
  A window. A door opened and a small \\
  Old \textit{bauren} in a coat stood there.  K. cried \\
  Out. "May I come in for a spell? I tried \\
  To reach the Palace, but I can not move \\
  Another foot!" A board began to slide \\
  Towards him and a few steps would remove \\
  Him from the snow and his plight would greatly improve.
\end{verse}
\begin{verse}
  \begin{center}
    \textbf{54} \\
  \end{center}
  At first, K. couldn't see shit. The large room \\
  Was dimly lit and K. soon stumbled o'er \\
  A washtub. A hand reached out from the gloom, \\
  (A woman's hand) to steady him, before \\
  K. could take a tumble down to the floor. \\
  From one corner came the screams of children \\
  And from another smoke billowed forth more \\
  And more, casting in a bewildering \\
  Way the light into a darkness quivering.
\end{verse}
\begin{verse}
  \begin{center}
    \textbf{55} \\
  \end{center}
  K. stood there dumbly as if in a cloud. \\
  "Who let the boozer in?" someone called out, \\
  "Who is he? Are you going to allow \\
  Anyone to wander in off the route?" \\
  "I am the land surveyor." K., without \\
  Being able to see anyone, said. \\
  "Oh hell, the surveyor." A voice about \\
  A few feet away said. K. turned his head, \\
  But still, he could not make out the living or dead.
\end{verse}
\newpage
\begin{verse}
  \begin{center}
    \textbf{56} \\
  \end{center}
  Finally, the room's smoke began to clear, \\
  And K. was able to make some things out. \\
  He saw that laundry was being washed near \\
  The door and a big wooden tub, about \\
  The size of two beds, was without a doubt \\
  The source of the smoke. Two men were bathing \\
  Together like a couple of fat trout \\
  In the steaming water. Pale snow-kissed rays \\
  Streamed in from a gap in the back, cutting the haze.
\end{verse}
\begin{verse}
  \begin{center}
    \textbf{57} \\
  \end{center}
  The light fell gently upon the figure \\
  Of a madonna with child at her breast. \\
  The pale light bathed her dress with a vigor \\
  That gave the garment a silken sheen blessed \\
  By the northen sunlight. What had so stressed \\
  K. for the past few days melted away. \\
  \textit{Bauren} children played near her, but at best, \\
  She was just merely of them, you could say, \\
  But she did not seem to belong to them that day. 
\end{verse}
\begin{verse}
  \begin{center}
    \textbf{58} \\
  \end{center}
  "Sit down!" came a shout from the wooden tub. \\
  A bushy-bearded soul gestured over \\
  The edge of the tub, splashing water up \\
  In K.'s face, leading him to discover \\
  The old man dozing without any cover \\
  On a chest in the corner of the room. \\
  K. desired a seat more than a lover \\
  Could desire their beau clothed in rich perfume. \\
  His weariness wrapped him in an oppressive gloom.
\end{verse}
\begin{verse}
  \begin{center}
    \textbf{59} \\
  \end{center}
  Grateful for a chance to finally sit, \\
  K. plopped down next to the napping old man. \\
  Now no one cared about him. The young fit \\
  Blonde woman at the washing trough began \\
  To sing softly as she worked, the kids ran \\
  Around the wooden tub as the men sent \\
  Potent splashes of water at the clan \\
  of urchins. The bath water did not dent \\
  K. one bit; he only thought of sleep heaven sent.
\end{verse}
\newpage
\begin{verse}
  \begin{center}
    \textbf{60} \\
  \end{center}
  Every epic demands a trip to sea. \\
  Paul was shipwrecked thrice and beaten with rods. \\
  A freakin' snake bit him and, like a boss, he \\
  Just tossed it off, I mean, what are the odds \\
  He was telling tales to impress the broads? \\
  Don Juan set sail for Cádiz before \\
  Finding his ship upended by the gods \\
  And cast adrift in the Aegean for \\
  Much longer than the Apostle Paul's day and more.
\end{verse}
\begin{verse}
  \begin{center}
    \textbf{61} \\
  \end{center}
  Oh, but K. is not destined for the sea! \\
  You see, he will not hear the sirens' call \\
  Or blind Polyphemus with a charred tree. \\
  Instead he stares at the woman in thrall \\
  To the armchair, appearing to be all \\
  But lifeless, not even seeing the child \\
  At her breast, vaguely staring at the wall \\
  Beyond. To K.'s eyes the sight appeared mild \\
  And ever unchanging, sad yet gracefully styled.
\end{verse}

\begin{center}
  \begin{tabular}{ c }
    \ccby \\
    This work is licensed under a \\ \href{https://creativecommons.org/licenses/by/4.0/deed.en}{Creative Commons Attribution 4.0 International License}.
  \end{tabular}
\end{center}  

\end{document}
